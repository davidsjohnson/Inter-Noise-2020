\section{Related Work}\label{sec:rw}

\subsection{Compressed Air Leak Detection}

The current state-of-the-art in compressed air leak detection is to use hand-held devices with highly directional ultrasonic microphone arrays for the detection and localization of air leaks \cite{Murvay2012:survey, Steckel2014:local}. Guenther and Kroll \cite{Guenther2016:ultrasonic} proposed an ultrasonic scanning system to automate the process with robotic systems.
Additionally, to reduce the amount of manual labor, Schenck et al. \cite{Schenck2019:airleakslam} proposed augmenting facility vehicles with LIDAR to combined simultaneous localization and mapping (SLAM) techniques with ultrasonic microphones.
Because ultrasonic sensors can be ineffective in noisy environments and require short distances from leak sources due to rapid attenuation of the audio signal, Erat and Meskell \cite{Eret2012:beamform} proposed a detection method using audible sound, and showed that beamforming with arrays of electret microphones is a feasible approach to leak detection and localization. 

While machine learning methods have shown success in other ISA tasks, the previous systems for air leak detection all used traditional signal processing techniques. Furthermore, there is only limited research on data driven leak detection with acoustic emissions. Desmet and Delore \cite{Desmet2017:ad} proposed a machine learning approach for anomaly detection to detect leaks in compressed air systems for mining equipment, but used air pressure levels for input. Kotani et al \cite{Kotani.2004} and Zhang et al. \cite{Zhang2003} proposed shallow neural network based detection for gas leaks using airborne sound. Our work builds on the idea of using audible sound for leak detection like Erat and Meskell \cite{Eret2012:beamform}, but with a data driven approach using deep learning.

\subsection{Industrial Sound Analysis}

The goal of ISA is to automatically analyze airborne sounds to automate production processes, such as predictive maintenance and quality control, that typically require manual labor or expensive sensors. The use of deep learning for ISA has been successfully applied to tasks such as monitoring the operational state of an electric engine, and classifying different bulk materials by their sound \cite{Grollmisch2019:isa, Grollmisch2020:embeddings}. Although successful, challenges remain for deep learning with ISA tasks, namely the need for large, robust datasets \cite{Johnson2020:robust}.

Popoular machine learning tasks, such as image recognition or natural language processing, benefit from large structured datasets often acquired through Internet sources, such as search engines and online review systems \cite{imagenet2009, Ni2019:amazon}. Developing large audio datasets, however, is more challenging due to a lack of inexpensive, easy to annotate data, especially in the case of industrial sounds. In recent years there has been an emergence of ISA datasets to foster further research.  Grollmisch et. al. \cite{Grollmisch2019:isa} published three datasets for common ISA tasks. Furthermore, two datasets were recently published for the detection of malfunction industrial machinery. ToyADMOS \cite{Koizumi2019:toyadmos} is a dataset of the operating sounds from a set of miniature machines, including anomalous sounds. Similiarly, the MIMII dataset \cite{Purohit2019:mimii} contains recordings from real production environments under normal and anomalous conditions. However, none of these datasets cover air leakage sounds. Therefore, we contribute a new ISA dataset for continuing research in compressed air leak detection using audible sound.
