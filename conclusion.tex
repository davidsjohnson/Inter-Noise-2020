\section{Conclusions}\label{sec:conc}

In this article, to foster further research in data driven approaches for compressed air leak detection we present a novel dataset composed of audio recordings in the audible frequency range of a compressed air network. The dataset contains audio recordings from four microphone locations over three recording sessions. For each recording session, recordings were captured with three leak states, each with five background noise conditions, for a total of 15 recording conditions and approximately six hours of audio per microphone. Furthermore, a baseline classification system using a CNN is proposed. To evaluate the viability of a data driven approach for leak detection using audible sound , five experiments using the proposed CNN are performed. 

The results show the potential of such a system even under noisy conditions. While background noise has an effect on model performance, promising results can be achieved when the classifier is trained using data containing background noise, even with different noise types in the evaluation data. Data augmentation slightly improves the performance of models trained on clean data, but further research could clearly boost the results. A possible solution could be to augment clean data with simulated industrial noise.

The effect of microphone placement on classification performance is also evaluated with our experiments. As expected the closest microphone performs the best. Interestingly, a non-directed microphone in the middle of the room shows good performance for detection, indicating that such a microphone may be able to capture leaks in unknown locations. This indicates an ability to strategically place a microphone in a general location around a compressed air network to identify locations with possible leaks for a manual inspection, without the need to survey the entire network. Different room sizes could be considered in further research. 

% One of the aims of compressed air leak detection is to help reduce need for manual efforts by experts. Using our proposed approached, a compressed air system could be augmented with microphones directed at critical junctions in the network to automatically identify when there is a leak without the need for an expert to survey the entire compressed air network. To make this feasible, more research is needed to determine if cheap microphones could be used instead of specialized measurement microphones. Additionally, this work shows the potential to implement a monitoring approach in which a microphone is placed in a more general location to detect leaks in a larger area with one microphone. Using this method, a maintenance engineer would be alerted to the general location of the leak and then make use of a manual device to find the exact location. This would eliminate (or reduce) the need for engineers to survey the entire compressed air network looking for leaks.

With this first work on a data driven approach to compressed air leak detection, we show that a CNN can detect air leaks in noisy condition with a high rate of accuracy. While there is room for improvement, such a system could save production facilities money by reducing manually efforts to find air leaks, affording for more frequent and continuous monitoring of compressed air systems.